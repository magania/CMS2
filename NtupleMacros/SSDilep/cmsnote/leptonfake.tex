\section{Data driven Method for Fake Lepton backgrounds (Type-II)}
\label{sec:leptonfake}

In this analysis, the primary source of events are from Type-II category. 
As shown in Table~\ref{tab:fakeOrigin}, roughly 3/4 of these events are expected to be from 
heavy flavor sources, while the remainder is from
fake leptons (Type-II b)). This means that one needs to be careful in defining the lepton fake rates
to make sure they predict both sources of fakes with sufficient accuracy.

In Reference~\cite{fakelep} we described a data-driven method to predict the fake 
background in a dilepton analysis. This method is being applied by us also 
in the WW~\cite{ww} as well as \ttbar ~\cite{ttbar} analyses. Here we briefly
summarize the method, and we apply it to the same sign dilepton study.

\subsection{The fake rate definition}
\label{subsec:fakeratedef}

The method starts by defining a fake rate ($FR$) measured in QCD events. We use 
the Pythia QCD samples with $\hat{P_T} > 30, 80 $ GeV. The fake rate is defined as 
the probability for a lepton passing loose cuts (Fakeable Object, $FO$) to pass the 
analysis cuts as a function of $p_T$ and $\eta$. The measured probability is then applied to 
dilepton candidates passing loose cuts to obtain a prediction to the fake lepton contribution. 
The details of the applications of the $FR$ are given in Section~\ref{subsec:fakerateapplication}. 

Fakeable Objects are defined as follows:

\begin{itemize}
\item Electron Fakeable Object, $eFO$:
\begin{itemize}
\item GSFElectron with $p_T > 10$ GeV;
\item $|\eta| < 2.4$;
\item No reconstructed muon with $\Delta R < 0.1$;
\item Electron ID and Conversion rejection defined in Section~\ref{sec:electron} as well as veto used in Section~\ref{sec:gsfctf};
\item Iso $<$ 0.4, where Iso=Sum/Max(20 GeV, $P_T$), and Sum = tkIso + hcalIso +  Max(0 GeV, ecalIso - 2 GeV).
\end{itemize} 

\item Muon Fakeable Object, $\mu FO$:
\begin{itemize}
\item Global and Tracker Muon with $p_T > 10$ GeV;
\item $|\eta| < 2.4$;
\item Global fit $\chi^2 /$ndof $ < 20 $;
\item $|d_0| < 200~\mu m$ (from silicon track, corrected for beamspot);
\item Iso $<$ 0.4, where Iso=Sum/Max(20 GeV, $P_T$), and Sum = tkIso + hcalIso +  ecalIso.
\end{itemize} 
\end{itemize} 

The $FR$ for electrons and muons are determined from the QCD sample.
% are shown in Figure.xx [Need a figure]
The $FR$ does not give a direct measure for an absolute lepton fake rate. It is the probability for a
fake lepton passing loose identification requirements as well as isolation to also pass a tighter
selection. 

\subsection{Application of lepton fake rate to our analysis}
\label{subsec:fakerateapplication}

We evaluate the $FR$ in \ttbar Monte Carlo events. This test is meant to check if the 
$FR$ as determined from the QCD events can be applied to \ttbar. In order to perform this test we 
define the following four event selections:

\begin{itemize}
\item \ttbar $\rightarrow Wb Wb \rightarrow \mu + e$  $ \nu \nu b b$:
\begin{itemize}
  \item Require a global muon with $p_T > 10$ GeV, truth matched to $W \rightarrow \mu$.
  \item Require a same sign electron that passes all the standard identification and isolation requirements. 
\end{itemize} 

\item \ttbar $\rightarrow Wb Wb \rightarrow \mu + (eFO \times FR)$  $ \nu \nu b b$:
\begin{itemize}
  \item Require a global muon with $p_T > 10$ GeV, truth matched to $W \rightarrow \mu$.
  \item Require a same sign $eFO$; weight each event by the $FR$ for the corresponding $eFO$.
\end{itemize} 

\item \ttbar $\rightarrow Wb Wb \rightarrow e + \mu  $  $ \nu \nu b b$:
\begin{itemize}
  \item Require an electron with $p_T > 10$ GeV, truth matched to $W \rightarrow e$.
  \item Require a same sign muon that passes all the standard identification and isolation requirements. 
\end{itemize} 

\item \ttbar $\rightarrow Wb Wb \rightarrow e + (\mu FO \times  FR)$  $ \nu \nu b b$:
\begin{itemize}
  \item Require an electron with $p_T > 10$ GeV, truth matched to $W \rightarrow e$.
  \item Require a same sign $\mu FO$; weight each event by the $FR$ for the corresponding $\mu FO$.
\end{itemize} 
\end{itemize} 

\begin{table}[hbt]
\begin{center}
\begin{tabular}{|l|c|}\hline
Sample & Event yield \\ \hline
\ttbar with $\mu + e$ (Observed) & 2.1 $\pm$ 0.3 \\
\ttbar with $\mu + (eFO \times FR)$ (Predicted) & 2.9 \\
\hline
\end{tabular}
\caption{ Monte Carlo test of the electron fake rate using 100 pb$^{-1}$ of integrated luminosity. \label{tab:EleFR_Test}}
\end{center}
\end{table}
\begin{table}[hbt]
\begin{center}
\begin{tabular}{|l|c|}\hline
Sample & Event yield \\ \hline
\ttbar with $e + \mu$ (Observed) & 2.4 $\pm$ 0.3 \\
\ttbar with $e + (\mu FO \times FR)$ (Predicted) & 2.7 \\
\hline
\end{tabular}
\caption{ Monte Carlo test of the muon fake rate using 100 pb$^{-1}$ of integrated luminosity. \label{tab:MuonFR_Test}}
\end{center}
\end{table}

The Monte Carlo test for the electron $FR$ consists of comparing event yields and distributions for 
$\mu + e $ and $\mu + (eFO \times FR)$. Similarly, for muon $FR$ it consists of comparing event yields 
and distributions for $ e + \mu$  and $e + (\mu DO \times FR)$. 
Results of the Monte Carlo tests for event yields are given in Tables~\ref{tab:EleFR_Test} and~\ref{tab:MuonFR_Test}.
The uncertainties are from the MC statistics. From these studies we conclude that the QCD $FR$ parametrization 
does a good job of reproducing the rate of fake electrons and muons in \ttbar events.

In order to obtain a prediction of the fake contribution to our analysis, we proceed in the following way:
\begin{itemize}
\item Select lepton $+ FO$ events where
\begin{itemize}
  \item one of the leptons passes all the standard identification and isolation requirements.
  \item the other lepton is a $FO$ but fails the standard identification and isolation requirements.
\end{itemize} 
\item The event passes all the standard kinematical cuts as outlined in Section~\ref{sec:eventselection}
\item Weigh each event by $FR/(1 - FR)$, where $FR$ is the fake rate for the $FO$ under consideration.
\item Add up all the weights.
\end{itemize} 

\vspace{2mm}
\begin{table}[hbt]
\begin{center}
\begin{tabular}{|l|c|c|c|c|c|c|}\hline
Same Sign leptons & Total &      Type-I &  Type-II & Type-II a) & Type-II b) & Type-III \\ \hline
$ee$ (predicted) &	0.21 &	0.01 &	0.20 &	0.15 &	0.05 &	0.00 \\
$\mu\mu$ (predicted) &	0.10 &	0.00 &	0.10 &	0.09 &	0.01 &	0.00 \\
$e\mu$ (predicted) &	0.31 &	0.01 &	0.30 &	0.26 &	0.04 &	0.00 \\
total (predicted) &	0.62 &	0.02 &	0.60 &	0.50 &	0.10 &	0.00 \\
\hline
\end{tabular}
\caption{ The number of events predicted using lepton fake rate method in \ttbar events for various types. 
Rates are normalized to 100 pb$^{-1}$.\label{tab:LeptonFakePredict}}
\end{center}
\end{table}
 The results of the application of the procedure outlined above is summarized in Table~\ref{tab:LeptonFakePredict}. 
We conclude the following in comparison with the observed events from Table~\ref{tab:fakeOrigin}:


\begin{itemize}
\item We predict within $\sim 20 \%$ of the observed Type-II contributions.
\item Within Type-II, the contribution from events with heavy flavor sources are largely predicted ($\sim 88 \%$).
\item The method introduces an overestimate for the true leptons in Type-I. This is at $\sim 2 \%$ level, 
this is negligible compared to the associated statistical as well as systematic uncertainties.
\end{itemize}
