\section{Application of the data driven methods to the SM and SUSY benchmark points}
\label{sec:application}
We apply the two data driven procedures to predict the backgrounds in the 
\ttbar-dominated SM sample. Table~\ref{tab:yieldsObsPre} shows the contribution 
of all SM background. The prediction and observation agree to within $\sim 30\%$.

\begin{table}[hbt]
\begin{center}
\begin{tabular}{|l|c|c|c|c|c|c|c|c|}\hline
Same Sign leptons & Total SM & \ttbar & tW & WZ & ZZ & WW & DY & Wjets \\ \hline
 $ee$ (observed) & 0.45 & 0.44 & 0.00 & 0.00 & 0.01 & 0.00 & 0.00 & 0.00 \\
 $ee$ (predicted) & 0.27 & 0.26 & 0.01 & 0.00 & 0.00 & 0.00 & 0.00 & 0.00 \\ \hline	
 $\mu\mu$ (observed) & 0.17 & 0.13 & 0.00 & 0.03 & 0.01 & 0.00 & 0.00 & 0.00 \\
 $\mu\mu$ (predicted) & 0.11 & 0.10 & 0.01 & 0.00 & 0.00 & 0.00 & 0.00 & 0.00 \\ \hline
 $e\mu$ (observed) & 0.48 & 0.39 & 0.04 & 0.04 & 0.01 & 0.00 & 0.00 & 0.00 \\
$e\mu$ (predicted) & 0.39 & 0.38 & 0.01 & 0.00 & 0.00 & 0.00 & 0.00 & 0.00 \\ \hline	
 total (observed) & 1.10 & 0.96 & 0.04 & 0.07 & 0.03 & 0.00 & 0.00 & 0.00 \\ 
total (predicted) & 0.77 & 0.74 & 0.03 & 0.00 & 0.00 & 0.00 & 0.00 & 0.00 \\ \hline
\end{tabular}
\caption{Observed and predicted  number of SM events passing the event selection in 100 pb$^{-1}$ of integrated
luminosity.\label{tab:yieldsObsPre}}
\end{center}
\end{table}

We also apply both of these methods to combination of SM and SUSY samples to derive the
prediction. Table~\ref{tab:yieldsSUSY}, show the contribution of observed and expected prediction to 
the sample. Typically one would compare observed with the prediction to look for excess 
in ``signal'' over the total background.
\begin{table}[hbt]
\begin{center}
\small\addtolength{\tabcolsep}{-5pt}
\begin{tabular}{|l|c|c|c|c|c|c|c|c|c|c|}\hline
Same Sign  & SM+LM0 & SM+LM1 & SM+LM2 & SM+LM3 & SM+LM4 & SM+LM5 & SM+LM6 & SM+LM7 & SM+LM8 & SM+LM9 \\ \hline
Observed & 45.54 & 10.02 & 2.09 & 7.55 & 3.4 &	1.79 &	2.86 &	2.01 &	4.32 &	3.50 \\ \hline
Predicted & 4.10 & 1.26 & 0.83 & 1.24 &	0.91 &	0.81 &	0.84 &	0.83 &	1.03 &	1.00 \\ \hline
\end{tabular}
\caption{Observed and predicted  number of SM and SUSY events passing the event selection in 100 pb$^{-1}$ of integrated
luminosity.\label{tab:yieldsSUSY}}
\end{center}
\end{table}
The associated systematic uncertainties are not discussed in this document. We plan to measure
them in data. Overall, the sources of systematics can be categorized as detector effects, effects 
of modeling of the contributing processes, uncertainties of the data-driven background 
prediction methods.  Systematic uncertainties from the lepton selection, ID, and reconstruction efficiencies will be
estimated based on the corresponding systematics of the tag-and-probe method used to determine these efficiencies 
$Z \rightarrow \ell \ell$ in data. We will assess the uncertainty arising from the jet energy scale
using $\gamma/Z$ balance with jets. The uncertainties of the data-driven background estimate will be studied using 
a measure of ``bias per lepton'' as a function of the either charge fake candidate in charge-flip rate or $FO$ in
lepton fake rate selections. The current document focuses on reducing and measuring SM backgrounds using the aforesaid
data driven methods.

\clearpage
